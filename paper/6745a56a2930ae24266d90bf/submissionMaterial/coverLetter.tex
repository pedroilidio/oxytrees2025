\documentclass{article}
\usepackage[utf8]{inputenc}
\usepackage{graphicx}
\usepackage{url}

\title{Cover Letter}
\author{Pedro Ilídio, Felipe Kenji Nakano, Alireza Gharahighehi, \\ Ricardo Cerri and Celine Vens}
\date{December 2024}
\begin{document}
%
\maketitle

% Both journals require authors to include an information sheet (for Machine learning submissions) or a cover letter (up to 2 pages) as a supplementary material (for Data Mining and Knowledge Discovery submissions) that contains a short summary of their contribution and specifically address the following questions: 

%     What is the main claim of the paper? Why is this an important contribution to the machine learning/data mining literature?
%     What is the evidence provided to support claims? Be precise.
%     Report 3-5 most closely related contributions in the past 7 years (authored by researchers outside the authors’ research group) and briefly state the relation of the submission to them.
%     Who are the most appropriate reviewers for the paper? Authors are required to suggest up to four candidate reviewers (especially if external to the Guest Editorial Board), including a brief motivation for each suggestion.
%     Optionally, list up to four researchers/potential reviewers with competing interests that should not be considered for reviewers.

Dear Editor,
~\\

With this letter, we are submitting our manuscript entitled ``Oxytrees: bipartite model trees for inductive interaction prediction" by Pedro Ilídio, Felipe Kenji Nakano, Alireza Gharahighehi, Ricardo Cerri and Celine Vens, which proposes a novel state-of-th-e-art tree ensemble model for bipartite learning. 
%and we would like to have our work considered for publication in the IEEE Access journal.
This manuscript is a significant extension of our previous conference paper presented at the 39th ACM/SIGAPP Symposium on Applied Computing. 2024. \cite{ilidio_fast_2024}. 


Bipartite learning is a machine learning task whose objective consists of modeling a function $(x_1,\; x_2) \mapsto y$ mapping a pair of objects $x_1$ and $x_2$ of different types to a binary output $y$ indicating the presence of interaction. For instance, bipartite learning is applied to predict interactions between proteins ($x_1$) and drugs ($x_2$).

In this work, we propose a state-of-the-art model trees ensemble, namely Oxytrees. More specifically, Oxytrees employs Kronecker regularized least squares (Kron-RLS) \cite{van_laarhoven_gaussian_2011}, a bipartite learning adaptation of the Ridge regression, as its leaf models. Furthermore, Oxytrees use novel impurity function and a novel inference algorithm, leading to a substantial improvement in terms of computational complexity


We evaluated Oxytrees on 15 datasets, from which 5 are proposed in this work, using different percentages of positive annotations masking, e.g., 25\%, 50\% and 75\%. In comparison to the current-state-of-the-art \cite{pliakos_drug-target_2020}, our results showcase that Oxytrees is mostly superior or competitive, while being X times more efficient 


We believe that our findings will appeal to your readership, since bipartite learning is an upcoming field in machine learning. Furthermore, your venue has recently published relevant paper on decision trees \cite{raymaekers2024fast, surer2024coefficient}, including the pioneer study on decision trees for bipartite learning  \cite{pliakos_global_2018}. 


Additionally, we also assume that future studies will refer to our manuscript since we propose a novel Python library with Scikit API,\texttt{bipartite\_learn}\footnote{\url{
https://github.com/pedroilidio/bipartite_learn}} that contains all datasets, methods and evaluation measured used in this work, thus allowing reproducible research on the field.

We confirm that this manuscript has not been published elsewhere and is not under consideration by another journal. All authors have approved the manuscript and agree with its submission to your venue.


~\\
Best regards,

Pedro Ilídio, Felipe Kenji Nakano, Alireza Gharahighehi, Ricardo Cerri and Celine Vens
\bibliographystyle{plain}
\bibliography{sn-bibliography}

\end{document}
