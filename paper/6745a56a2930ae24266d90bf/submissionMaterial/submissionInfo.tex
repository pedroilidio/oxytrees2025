\documentclass[sn-mathphys-num]{sn-jnl}

\begin{document}
\title[Oxytrees]{Oxytrees: model trees for bipartite learning}
1 -  Keywords

Interaction prediction, model trees, positive-unlabeled learning, Kronecker regularized least squares,  

2 -  Journal
Machine Learning


3 -  3. What is the main claim of the paper? Why is this an important contribution to the machine learning literature?  
In our work, we claim that our proposed model tree method, Oxytrees, is competitive or superior in terms of performance, while being substantially faster than the current state-of-the-art. Precisely, our method can improve the state-of-the-art in interaction prediction.

Despite widely investigated, very few methods in the literature can predict new interactions in any interaction problem, as they are either application specific, as often seen in deep-learning based methods, or must be completely re-trained to predict new interaction, as seen in matrix factorization methods. The current state-of-the-art, an ensemble of bi-clustering trees, is applicable to any interaction problem, nonetheless it does not scale well due to the computational complexity of its induction and inference algorithms. 

Our work addresses these challenges by proposing a novel state-of-the-art method that is superior or competitive in terms of performance, while being computationally faster both in terms of induction and inference. 


4 - What is the evidence you provide to support your claim? Be precise
To the best of our knowledge, we present the largest comparison on interaction prediction in the literature. Precisely, we performed X-fold cross validation on 15 datasets and X methods, including the current state-of-the-art and baselines methods based  on the global multi-output (GMO) and local multi-output (LMO) approaches. Further, we also performed experiments where a percentage of the positive annotations were masked to simulate positive-unlabeled learning, namely 25\%, 50\% and 75\%.  Our results reveal that our proposed methods are superior or competitive in terms of predictive performance, specially in cases where most of the annotations are masked. Precisely, (ADD numbers to show how better we are) AND MENTION STATISTICAL TESTS

5 - What papers by other authors make the most closely related contributions, and how is your paper related to them? * 

The studies of Pliakos et al. [1-4] are the most relevant works since they are, to the best of our knowledge, the current state-of-the-art. As further discussed in the manuscript, we improved on them, both in terms of performance and computational time, by employing model trees, proposing novel induction and inference algorithms and using a more extensive evaluation setup, which include more methods and datasets. 

The method NRLMF [5] is also closely related, since it is used by our proposed method, and it is, to the best of our knowledge, the only matrix factorization method that supports inductive inference. 

Similarly, Kronecker regularized least squares (KRON-RLS) [6] is also a component of our method and an overall strong comparison method in interaction prediction.

[1] Pliakos, Konstantinos, and Celine Vens. "Drug-target interaction prediction with tree-ensemble learning and output space reconstruction." BMC bioinformatics 21 (2020): 1-11.
[2] Pliakos, Konstantinos, Celine Vens, and Grigorios Tsoumakas. ``Predicting drug-target interactions with multi-label classification and label partitioning." IEEE/ACM transactions on computational biology and bioinformatics 18.4 (2019): 1596-1607.
[3] Pliakos, Konstantinos, and Celine Vens. ``Network inference with ensembles of bi-clustering trees." BMC bioinformatics 20 (2019): 1-12.
[4] Pliakos, Konstantinos, Pierre Geurts, and Celine Vens. ``Global multi-output decision trees for interaction prediction." Machine Learning 107 (2018): 1257-1281.
[5] Liu, Yong, et al. ``Neighborhood regularized logistic matrix factorization for drug-target interaction prediction." PLoS computational biology 12.2 (2016): e1004760.
[6] Van Laarhoven, Twan, Sander B. Nabuurs, and Elena Marchiori. ``Gaussian interaction profile kernels for predicting drug–target interaction." Bioinformatics 27.21 (2011): 3036-3043.

6 - Have you published parts of your paper before, for instance in a conference? If so, give details of your previous paper(s) and a precise statement detailing how your paper provides a significant contribution beyond the previous paper(s)
A preliminary version of our work has been published in [6] where the induction algorithm used by our method was proposed. In this manuscript, we extend our previous work in X steps: i) we propose model trees which employs Kronecker regularized least squares as its models at the leaves; ii) a novel considerable faster inference algorithm, which can be used by any decision tree in interaction prediction; iii) a theoretical analysis of the induction algorithm previously published [6]; iv) a more extensive evaluation setup where 5 extra datasets were added; v) A novel Python library with Scikit API for interaction prediction which includes all methods, datasets and evaluation measures employed in this work, allowing reproducible research in the field.

[6] Ilídio, Pedro, André Alves, and Ricardo Cerri. ``Fast Bipartite Forests for Semi-supervised Interaction Prediction." Proceedings of the 39th ACM/SIGAPP Symposium on Applied Computing. 2024.
\end{document}